Broad exploration is often the best way to solve complex problems because the best solution might not come to mind first. Consequently, creative thinking, like simulated annealing, begins with a “hot” exploratory phase of broad solution search and gradually “cools” to exploitation and narrowing the search space. In solving complex, open-ended problems, because the best approach may not come to mind first, broad exploration is important. However, a common cognitive bias is to ``cool'' too soon, settling on the first adequate idea that comes to mind or fixating on low-level details, leading to under-exploration of potentially better ideas. The challenge is knowing both what and how to explore, seeing the metaphorical forest for the trees. This dissertation presents how attuning people to a context-appropriate level of detail at the right time will catalyze creativity. This dissertation addresses three core questions for tackling complex problems: ``How do I get started?'', ``how do I accomplish my goal?'', and ``how do I move forward?'' All three of these questions stymie problem-solving--especially when ``better'' is multifaceted or ambiguous-- and novices especially often fixate on whether an idea is ``correct.'' This dissertation presents interactive system designs that help people surmount these three creative challenges.

This dissertation introduces three exploratory strategies and demonstrates their efficacy. First, abstraction blocks scaffold tangible chunking and visual exploration of ideas to help novices get started in flexibly examining alternatives. Second, adaptive conceptual guidance uses domain-specific heuristics to adaptively present examples depending on a person’s progress and task to help them better understand and apply concepts to accomplish their creative goals. Last, interactive feedback reuse combines structural guidance and adaptive suggestions of previously-given expert feedback to help improve critique and move forward on creative work. These strategies and their evaluations demonstrate that structuring exploration and providing adaptive examples in the context of people’s work can help people discover otherwise unseen possibilities and better achieve their creative vision.


