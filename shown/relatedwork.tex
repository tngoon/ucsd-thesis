

% move following section to chapter 2
\subsection{Examples as Reflective Dialogue}
Sch{\"o}n \cite{schon1984reflective} presents an architecture studio scenario where a teacher guides a student how to reframe the given problem, using example sketches to demonstrate key concepts tailored to where the student needs the greatest assistance. The expert in this scenario uses examples to support concepts that fit the novice's situation and where they need the most help. Presenting the right concepts and supporting examples at the right time can lead to a reflective dialogue between a creator and their work in which different alternatives are explored and evaluated for their relative benefits. Similarly, human tutoring studies show how tutors use interactive dialogue to identify knowledge gaps and tailor using examples to help learners better understand problems \cite{Chi2001,mclaren2008}. Adaptively using examples as a form of support can not only help novices solve a specific problem, but also understand the conceptual reasoning underlying the example.

\subsection{Computational Aid for Applying Examples}
Given these challenges of examples presentation and use, how might creativity support tools better help novices understand how to implement the ideas that examples can provide?
Computationally-generated guidance can take a contextual approach, presenting examples tailored to a user's current task or creative intent \cite{fraser2019replay,kandel2011wrangler,kumar2011, Lee2010}. Two exemplar systems demonstrate this approach for both visual and text-based work. CritiqueKit uses past expert critique as examples and suggests these examples as a reviewer gives feedback based on whether the feedback fits certain criteria \cite{ngoon2018interactive}. This approach uses text-based context to determine whether feedback fits structural characteristics to adapt what type of examples are most helpful for a novice reviewer. In visual work, adaptive templates for photography suggest gridlines to help novice photographers apply composition concepts to their photos, using visual context as a basis for suggesting composition options \cite{jane2020adaptive}. We extend computational guidance approaches by adapting not only the examples shown but also the \textit{concepts} they embody to help novices see past the details and better understand and apply the insights within examples.