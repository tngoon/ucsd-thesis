\section{Harnessing the Power of Examples in Creative Work}

A perennial challenge for novices is figuring out how to get started on a creative endeavor. Viewing and adapting others' work can help novices find traction in a mountain of possibility. Good examples can give inspiration, unveil new ideas, and highlight the advantages of alternative paths \cite{chan2011benefits,Siangliulue}. Examples can also help people compare and contrast different ideas across examples, drawing attention to important features (such as the composition or color scheme used by a painting) that a creator might apply to their own work \cite{alfieri2013learning,chi2012seeing,gentner2003learning,kulkarni2012early,marsh1996examples}. 

Today, with the proliferation of online communities and galleries, real examples of creative work are highly accessible.
Communities like Behance (www.behance.net) and HitRecord (www.hitrecord.org) present examples in searchable galleries that enable users to browse, modify, and implement ideas from creators with diverse backgrounds and levels of expertise, bolstering exploration and inspiration \cite{kang2018paragon,Lee2009,Lee2010}. Another strategy for supporting novices is to present examples within tools. Tools such as iMovie (www.apple.com/imovie) and Adobe Spark (www.spark.adobe.com) leverage expertise through provided templates or expert patterns. Novices in particular benefit from templates because they can shape their ideas and work to the high-level structure provided by the template \cite{Kim, kumar2011, yuan2016}.
For more procedural tasks, video tutorials serve as accessible examples that give guidance and important factual information for achieving concrete goals \cite{van2003} such as knitting a tri-colored scarf or sketching a human face. Presenting examples enables people to follow in the footsteps of others, giving a point of comparison and achievement for their own work.

\subsection{The Challenges of Applying Examples}
In order to use examples effectively, users must be able to assess the context and concept of the suggested example being presented. For example, if a novice photographer sees an example of a sunset photo with an off-center composition, they may not understand how the framing or composition affects the overall photo quality.  
Presenting examples shares similarities to search interfaces where a primary challenge is how to select and present the most relevant example or result \cite{hearst2009search,matejka2011ambient}. 
Even when examples are present and available, novices especially may not understand what examples are most relevant or how to implement the ideas within them, instead replicating surface details \cite{javadi2012impact}. Without requisite domain knowledge, discerning how to build upon examples without simply replicating them can be difficult.  

While examples can give inspiration, they can also paradoxically constrain creativity. Examples that are too semantically different from the target goal might actually be harmful for ideation \cite{chan2011benefits,chan2017semantically}.  Timing of examples also matters; examples presented too late are no more beneficial for creative outcomes than not viewing examples at all \cite{kulkarni2012early,Siangliulue}. Additionally, when given specific examples, novices tend to conform to salient surface features of examples and apply them to their own outcomes, causing potential fixation and decreasing novelty of ideas \cite{jansson1991design,kulkarni2012early,marsh1996examples,Smith1993}. Similar to work comparing experts and novices, novices tend to focus on the details of examples rather than on the underlying concepts \cite{chi1981categorization}. An important consideration for presenting examples is choosing the right examples at the right moments.

% inline websites (get rid of the https)

\subsection{Adaptive Conceptual Guidance: Examples in Context}
This chapter investigates \emph{adaptive conceptual guidance}, which presents in-situ examples and helps users apply insights from them. We show high-level domain principles at specific points in time as a user works to suggest relevant options that they may not notice or consider. In contrast to providing a library of examples or providing templates \cite{kang2018paragon,Lee2009,marks1997design,terry2002side}, this approach contextualizes examples based on the user's activity to give specific help at the right moments, which we hypothesize will aid example use and creative outcomes. 

We investigate this hypothesis through implementing a Wizard-of-Oz prototype called \emph{Sh{\"o}wn} that uses adaptive conceptual guidance to help novices explore and apply examples in the domain of comic drawing.
We choose comics as a domain because it involves multimedia storytelling and does not require special equipment, making it a useful exemplary domain for studying the challenges related to creative decision-making when multiple media (such as writing, drawing, and story pacing) are involved. 
To unearth concrete novice and expert differences for this domain, we conducted participant observations with nine expert comic artists and nine novices. We found that novices, in contrast to experts, focus on low-level detail when asked to create a comic strip and struggle to explore alternative compositions and story ideas.

We evaluated adaptive conceptual guidance in a between-subjects experiment where participants created a comic strip. This experiment ($n=24$) compared comics created by participants using one of two versions of Sh{\"o}wn:
a) a version that provided examples for users to browse on their own and b) a version that provided adaptive conceptual guidance and examples to illustrate possible ways of framing or composing comic panels. Adaptive conceptual guidance enabled participants to create comics that were rated by readers as having more unique and clear stories and drawings than comics created without adaptive guidance (Figure \ref{fig:woz}). Raters also preferred comics created with adaptive conceptual guidance overall. Participants stated in interviews that guidance inspired them to consider different ways of showing their story, lending support to the hypothesis that adaptive conceptual guidance assists exploration. Most participants without adaptive conceptual guidance either did not view the examples or did not find them useful, further supporting the hypothesis that novices need guidance for knowing how and when to implement examples to their own work. 

% To summarize, this chapter contributes the following:
% \begin{itemize}
%     \item an interview study that demonstrates how novices focus on details and struggle with exploring alternative ways to show their comic story,
%     \item a Wizard-of-Oz system that contextualizes and adaptively suggests examples and concepts at particular moments based on a user’s activity, and 
%     \item empirical evidence that adaptive conceptual guidance can help novices better utilize examples to improve their creative outcomes.
% \end{itemize}