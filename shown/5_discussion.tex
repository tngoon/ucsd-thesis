\section{Discussion \& Future Work}
This chapter investigated whether providing adaptive conceptual guidance aids example use more effectively than non-adaptive examples through a Wizard-of-Oz prototype and empirical study. We next discuss implications of our findings for future creativity support tools and generalization to other domains. 

\subsection{System-Directed Versus User-Directed Creativity Support}
In our experiment, we observed that novices in both conditions rarely sought examples without being prompted, perhaps because they were unaware of needing help or did not know what help they needed. Only two \textit{adaptive} condition participants proactively asked for guidance, and only three \textit{non-adaptive} condition participants viewed the examples gallery on their own. While exploring alternatives is a useful problem-solving strategy, meaningfully evaluating them requires expertise and the comparison induces a hefty cognitive load \cite{tuovinen1999comparison}. In line with prior work \cite{bilalic2008good, Graesser1994,Siangliulue}, this suggests that timing is relevant to giving the most appropriate help as novices may not know what type of help is most helpful and when. By incorporating concepts with examples, we found that Sh{\"o}wn's real-time adaptive conceptual guidance helped to expand participants' field-of-view to concepts they would not have known to explore on their own. Such proactive system-directed approaches seem to benefit open-ended creative problems (like figuring out how to draw a comic story) because they help users explore beyond known possibilities.

Similar to other systems that present galleries of examples to provide broad inspiration \cite{kang2018paragon,Lee2009,marks1997design}, Sh{\"o}wn provides high-level examples meant for conceptual exploration rather than examples that are particular to the user's specific work. For example, if a user wants to draw two people talking, Sh{\"o}wn presents general examples of different panel framings or combinations of text and images rather than specific instances of two people talking in a panel. These context-agnostic examples are framed as considerations (rather than statements that impose specific approaches as the ``correct'' ones). This open-ended approach is analogous to human tutoring, where less didactic interactions allow the learner to form their own explanations and hypotheses in evaluating ideas \cite{Chi2010}. 
In addition, recommending guidance in-situ allowed participants to better understand the relevance of guidance and evaluate examples in context \cite{schon1984reflective}. One participant explicitly mentioned that showing guidance in the moment while drawing encouraged them to view the examples when they would not have otherwise. Participants cited that even if they did not change their initial idea, the provided examples served as inspiration or as a way to help them explain their choices. In these instances adaptive conceptual guidance seemed to serve as a nudge for users to consider different options or evaluate their own decisions. 

In contrast, user-directed approaches (such as query-based systems and other help-seeking tools \cite{fraser2019replay,Huang2019, Satyanarayan2014}) seem to work best when the user has a concrete idea they want to execute. We saw this with how participants made use of Sh{\"o}wn's drawing helper, which produced basic icons to quickly visualize ideas based on a user's request. Some used voice commands to direct the drawing helper to provide images. However, the drawing helper had its limits: it could not generate drawings in response to complex, ambiguous requests. For example, P15 (\textit{non-adaptive}) asked the drawing helper to lay out an entire panel: ``\textit{Show me a man holding a phone with a dating app on it and lists for interests.}’’ 
This participant attempted to shape their ideas around what the drawing helper could generate, rather than using it to show their own ideas. This reflects how novices may form an over-reliance on help-tools or conform to salient features rather than high-level considerations \cite{jansson1991design,javadi2012impact,marsh1996examples,Smith1993}. Our findings with Sh{\"o}wn suggest that user-directed guidance works best for small and concrete tasks where the user already has an idea in mind.

Both system-directed and user-directed guidance can support the creative process in complementary ways. Context-agnostic conceptual aid allows for the ambiguity of exploration while specific user-directed guidance can give a concrete direction towards accomplishing goals. In many ways, using examples as guidance is similar to coaching or tutoring, where a coach or tutor gives both open-ended hints or suggestions and concrete examples depending on a person's progress. Sh{\"o}wn provides a demonstration of using context from user actions to leverage such guidance for creative work. Future research could further examine how computational systems can provide more adaptive coaching or tutoring for creative endeavors.

% limitations of current study
\subsection{Designing Adaptive Conceptual Guidance Systems: Limitations \& Opportunities}
We found two areas where adaptive conceptual guidance could be improved. The first is in determining the timing of guidance. One participant felt that guidance was presented too late and would be more helpful or more relevant earlier. While guidance was shown adaptively, the heuristics for determining when to show certain concepts and examples were static for each participant, which is both a limitation of the Wizard-of-Oz implementation as well as potential technical implementations. One open question is how to better anticipate when to provide guidance when even the user does not know when to ask for help. Potential effective timing mechanisms might be measuring how long a user is idle \cite{chan2018best,Siangliulue} or taking a mixed-initiative approach in incorporating user input, such as refining or pruning suggestions \cite{kandel2011wrangler} to determine the best moments to present examples. 

The second area of improvement is providing transparency around why guidance was being given. Three participants wanted further explanation for why certain pieces of guidance were being shown at specific times. One participant asked the experimenter whether the system presented guidance randomly or not. Another thought the presentation of guidance meant the system did not ``like'' their current drawing. This sentiment reflects prior work showing that people prefer explainable and transparent interactions with AI systems so they not only understand what help is being given but also why \cite{Amershi,Heer2019,Oh2018}. One possibility is to display guidance alongside interactive checkboxes or dynamic rubrics to help users understand how to better situate guidance within the user’s context \cite{Bharadwaj,ngoon2018interactive}.

We also found potential in improving user-directed support. While participants did not often explicitly seek out examples while working on their comics, many participants in both our interviews and experiment asked ideation questions aloud such as ``\textit{How do I show this place is empty?}'' or ``\textit{How do I show that this character is angry?}'' as part of the think-aloud protocol. These questions could serve as queries to indicate user intent and aid in tailoring the type of examples shown. During exploratory tasks like brainstorming ideas or deliberating ways to translate an idea visually, users may have more of these open-ended queries in mind rather than concrete goals. Tools that adaptively display help for just-in-time learning as well as incorporate contextual search mechanisms like natural language and deictic instructions may be useful ways of enabling user-directed conceptual support \cite{fraser2020remap, Graesser2001, laput2013pixeltone, Yoon}. 

%limitations of system design
As a Wizard-of-Oz prototype, Sh{\"o}wn utilizes a human Wizard to use a user's actions as heuristics to determine what conceptual guidance to provide. Some of the heuristics were based simply on starting on new panels (\textit{i.e.} the Wizard shows the guidance to consider moments or transitions when the user moves on to a new panel). Other heuristics were based on capturing the user's past and current actions (\textit{i.e.} the Wizard presents guidance on images and words if the panel contained too much or too little text). Existing tools already use sketch-based actions as context such as Procreate's Quick Shape tool\footnote{https://procreate.art/handbook/procreate/guides/quickshape//} to automatically complete shapes or straighten a user's lines or Google Jamboard's Autodraw tool that uses object recognition to infer what icons to suggest to the user. Co-creative intelligent agents can also use object and line recognition to improvise collaborative drawing \cite{Davis2016}. Our Wizard-of-Oz evaluation shows how adaptive systems might use visual, sketch-based context to provide conceptual guidance beyond automated drawing help or example galleries. Open remaining questions are how systems might be trained to learn such context as well as what types of context and concepts are most appropriate for adaptive and automated assistance beyond sketching.

One limitation of Sh{\"o}wn was that the example screens were separate from the drawing screens, requiring users to switch between drawing and example screens in order to view them. Tools incorporating adaptive conceptual guidance or examples presentation should consider more ambient displays that show examples without requiring the user to change the context of their work. Because of the remote nature of the study, one limitation was that we could not control for the screen size participants used. This may have affected drawing ability, though this limitation may have been mitigated since all participants used a stylus and had the drawing helper available.

Tools that help novices overcome skill barriers to execute their creative goals are powerful, but without seeing potential alternatives and possibilities, novices remain bounded by their limited conceptual expertise. Sh{\"o}wn evaluates simple recognition heuristics through a Wizard-of-Oz implementation to contextually present concepts and examples for better exploration and execution. Future work could examine feasibility and implementation of such heuristics for creativity systems across domains. We find that combining user-led support for concrete queries and adaptive support for open-ended exploration of alternatives helps novices better understand and utilize alternatives for their own work. An interleaving of system-led support and user-led agency is a promising direction for the development and evaluation of future creative tools \cite{Heer2019}. We provide a demonstration of this direction that can apply to creative endeavors across domains.

\subsubsection{Summary}
% do this for all chapters
We present Sh{\"o}wn, a Wizard-of-Oz system that provides adaptive conceptual guidance by suggesting relevant concepts and examples for a user to explore based on their current task. We hypothesized that adaptive conceptual guidance would guide the application of examples and improve creative work more than simply providing static examples alone. Through interviews and a between-subjects experiment in the domain of comics, we found that adaptive conceptual guidance led to comics with more unique and clear stories and clearer drawings. Users also found adaptive conceptual guidance to be timely and useful for inspiration while users without guidance did not find the examples as relevant and were less likely to take advantage of them. We argue for tools that help novices explore ideas by providing the right kinds of assistance in the right situations. In this way, we may better expand novices' vantage points to explore concepts more broadly and improve creative work. This chapter provides a direction for the future of adaptive creativity support tools and presentation of examples.

\subsubsection{Acknowledgements}
We thank the novice and expert comic artists for participating in interviews and our experiment participants for their time and efforts. This research was funded in part by Adobe Research.

This chapter, in part, includes  portions of material as it appears in \textit{Sh\"{o}wn: Adaptive Conceptual Guidance Aids Example Use in Creative Tasks} by Tricia  J. Ngoon, Joy O. Kim, and Scott Klemmer in the Proceedings of the 2021 ACM Conference on Designing Interactive Systems (DIS '21). The dissertation author was the primary investigator and author of this paper.