Six years and more than ten existential crises later, I've finally finished this thing that we call a PhD. Doing a PhD is not an easy journey, especially during a global pandemic. I certainly could not have achieved this feat without the help of so many amazing people in my life. This is my attempt to show my immense appreciation and gratitude to all those who have led me to this accomplishment.
 
Thank you to my committee -- Scott, Caren, Joy, Steven, Bill, and Philip -- for your endless support and feedback. In particular, thank you Scott for being my biggest advocate and pushing me to be a better researcher and thinker. I never would have thought that taking your Interaction Design course on Coursera would lead to me applying to UCSD almost on a whim. I am so grateful that you saw and continue to see potential in me. Thank you to Caren for being such a big influence in my research and a true inspiration. Thank you Joy for being an amazing mentor and welcoming me into the Adobe Research family. Thank you Philip for always checking in and providing me with encouragement. Thank you Steven for the fruitful research discussions and feedback. Thank you Bill for your stimulating discussions and questions during lab meeting and your constant encouragement. 
 
I am blessed to have incredible research collaborators, colleagues, and research assistants. Thank you Adobe Research for giving me opportunities to expand my research, and especially to Mira Dontcheva for all your wonderful feedback and advice through our collaborations. Thank you Michelle Lee, Nicolas La Polla, and Vivian Leung for being such bright and hard-working RAs. This research would not have been possible without your dedication. I am especially proud of Vivian for completing a wonderfully written undergrad thesis on our work together. Thank you Design Lab grad students for creating a supportive and collegial environment within the lab. Thank you to the extended HCI research community and all the people I have had the pleasure of meeting at conferences. In particular, thank you Eunice for being an awesome C\&C Student Volunteer Co-Chair and allowing me to share your hotel room when I was stranded in Austin during CSCW. Thank you Yea-Seul for being a fellow ``Rising Star'' with me and extensive discussions about our futures after grad school. Thank you Grace for all the insightful conversations interning together at Adobe. Thank you Minsuk for always being a friendly face at conferences. Thank you Jane for all the productive ``work’’ sessions. You have truly been a positive influence on my research and mental health, and I'm a little sad that we won't get to collaborate together at UCSD. Going further back, I am thankful for Tanya Evans’s and Christian Batista's mentorship during my time at Stanford. My interest in research all stemmed from Art Shimamura's Learning \& Memory class, and I am grateful for his guidance in completing my senior thesis with him.
 
I am also grateful for the Design Lab and the Cognitive Science Department. Thank you Olga, Sara, Vanessa, Ian, and Michel\'{e} for welcoming me into the Design Lab with open arms. Thank you Teenah Eco for being the first person I met in the lab and always being a smiling and friendly presence. Thank you Colleen for frequently checking in on how my family was doing. Thank you to Don Norman for being supportive of everyone in the lab and providing insightful feedback throughout my time here. Thank you Beverley for being the British grandmother I never had and always lending a supportive ear. Thank you Ethel for all your help in getting me to the finish line. Thank you Thanh for supporting me and the rest of the department through all your hard work. Thank you Marta Kutas and Andrea Chiba for helping build my formative research years through my 2nd and 3rd year projects. Thank you Drew Walker for being an advocate for TAs and helping me grow as an instructor. Thank you Doug Nitz and Eran Mukamel for standing up for CogSci grad students when we needed it most.
 
A huge amount of gratitude goes out to my support crew of friends. Thank you to the crewtons for all the sushiboocha and other shenanigans. Thank you to my tiny, but mighty MARKETS cohort (Michael, Amy, Reina, Kevin, Eric, and Shuai) for having each others' backs all these years. Amy, I remember being roommates with you when we first interviewed at UCSD, and I'm so glad we both ended up here and remained such close friends. Ailie, my two-time award-winning co-author, awkward travel buddy, and awesome friend, thank you for all your support both in research and general life. Vineet, thank you for being a shining example of passionate, hard work and for sending your wonderful Friday clips. Thank you to all my past officemates (Ariel, Julia, Sam, Srishti, Matin, Nida, and Yasmine) for making our tiny closet office more bearable. Thank you Ariana for being such an understanding and strong person in my life and pushing me to be more assertive and social. Thank you Janet for being a fun adventure friend and discussing all things Marvel with me. Thank you Chris for being a fellow awkward friend even though I yell at you a lot (deservedly). To my roommate Rob, thank you for making life infinitely more loud and interesting. Robercia Junction will live on forever. Thank you Vicky, Kenny, and Zoe for being great travel buddies and for all the fun movie nights. Thank you Gio for inspiring a lot of my thoughts about coaching and creativity. Thank you Jocelyn for remaining one of my closest friends and a constant source of positivity over the years. Thank you Calvin for remaining a supportive figure in my life even at a distance and being my sounding board for ranting. Thank you to my extended Wushu family who I know I can always count on even if we're worlds apart. Thank you to the rest of the Fab 4 (Paola, Sandhya, and Sally) for your inspiration and friendship. You are all doing such amazing things. 
 
Finally, I would like to give the biggest thank you to my extended and immediate family for their unconditional love and support. To Lynn, you are more like a sister than a cousin to me. We are the same type of weird, and I am thankful we've been so close for our entire lives. To George, you have always been family to me. You are now stuck with putting up with Lynn and me together forever. To Jessica, we don't see each other often, but I'm proud of the people we have both become. To my big brother Chris, I've truly enjoyed seeing our relationship improve to where we can watch game shows, travel, and laugh at a hundred inside jokes together. To my mother and father Josephine and Peter, thank you for always being a strong influence in my success. This dissertation is dedicated to you, and I hope to continue to make you proud.
\\

\textsc{Chapter \ref{chapter:abstraction}}, in part, is currently being prepared for submission for publication of the material by Tricia J. Ngoon, Joy O. Kim, and Scott Klemmer. The dissertation author was the primary investigator and author of this material.

\textsc{Chapter \ref{chapter:shown}}, in part, includes  portions of material as it appears in \textit{Sh\"{o}wn: Adaptive Conceptual Guidance Aids Example Use in Creative Tasks} by Tricia  J. Ngoon, Joy O. Kim, and Scott Klemmer in the Proceedings of the 2021 ACM Conference on Designing Interactive Systems (DIS '21). The dissertation author was the primary investigator and author of this paper.

\textsc{Chapter \ref{chapter:critiquekit}}, in part, includes portions of material as it appears in \textit{Interactive Guidance Techniques for Improving Creative Feedback} by Tricia J. Ngoon, C. Ailie Fraser, Ariel S. Weingarten, Mira Dontcheva, and Scott Klemmer in the Proceedings of the 2018 ACM Conference on Human Factors in Computing Systems (CHI '18). The dissertation author was one of the primary investigators and authors of this paper.

\textsc{Chapter \ref{chapter:future}}, in part, is currently being prepared for submission for publication of the material by Tricia J. Ngoon, Vivian Leung, and Caren M. Walker. The dissertation author was the primary investigator and author of this material.

\textsc{Chapter \ref{chapter:future}}, in part, includes portions of material as it appears in \textit{The Dark Side of Satisficing: Setting the Temperature of Creative Thinking} by Tricia J. Ngoon, Caren M. Walker, and Scott Klemmer in the Proceedings of the 2019 ACM Conference on Creativity and Cognition (C\&C '19). The dissertation author was the primary investigator and author of this material.

% \textsc{Chapter \ref{chapter:future}}, in part, includes portions of material as it appears in the paper in submission \textit{Constructive Activities for Supporting People's Creative Scientific Insights} by Vineet Pandey, Tricia J. Ngoon, and Samuel Lau. The dissertation author is an author of this paper.