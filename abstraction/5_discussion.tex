\section{Discussion \& Future Work}
We investigated how directly manipulable abstract blocks affect communication in collaborative sketching. In an observational design study, we found that abstract blocks gave groups a way to break down drawing compositions through movable, simple shapes that de-emphasized details and permanence. They allowed \textit{abstraction} groups to communicate higher-level concepts and explore different ideas and compositions relative to \textit{freeform} groups, which used planning phases to discuss more superficial details rather than compositional concepts. In this section, we discuss the implications of these findings for future creativity support tools.

\subsection{Tools for Supporting Abstract Blocking}
This chapter investigated how scaffolding abstraction through abstract blocks affects exploration and communication in the task of collaborative drawing. 
In any collaborative task, communication can leave ephemeral, intangible artifacts that aid groups in coordinating actions and developing rapport \cite{Davis2017,Davis2016}. For example, group dynamic and physical gestures are important for ideation and building consensus in collaborative drawing tasks \cite{Bly1988, Tang1991}. In our observations, groups used both verbal and non-verbal forms of communication (like gesturing) to make decisions and share ideas \cite{goldin1999}. However, for \textit{abstraction} groups, abstract blocks helped participants concretely express intangible ideas through the placement of directly manipulable shapes, giving groups a malleable visualization of a concept that groups could modify as they conversed. The tangibility of making abstract concepts feel more concrete and importantly, changeable, seemed to be the primary benefit of abstraction blocks.

Groups could, of course, use blocking and similar abstraction strategies without a need for physical blocking tools. However, we observed in our study that \textit{freeform} groups did not use such strategies with the exception of one group who had prior experience in composition concepts through photography. Though sketches are meant to be blocking tools themselves, novices especially do not often separate exploratory sketching from refined drawing \cite{welch2000sketching}, as we observed in \textit{freeform} groups. When a concrete abstraction tool---abstract blocks---was made available, \textit{abstraction} groups were more inclined to create abstract representations. Some groups chose to use text to label blocks, others chose to lightly sketch, but importantly, \textit{abstraction} groups did not decorate blocks with heavy detail. Abstract blocks may have helped form common understanding of the drawing among collaborators \cite{olson2000distance}, helping groups maintain discourse on higher-level concepts.

Abstraction through abstract blocks may be even more beneficial in online settings where intangible communication is difficult. In such settings, groups must rely on technological affordances to effectively collaborate \cite{Hollan1992,Jensen2018}. 
Future collaborative tools could explicitly encourage abstract blocking, as opposed to requiring creators to figure out how to create abstract representations of their work themselves through existing tools (\textit{e.g.}, shape tools in Microsoft PowerPoint or Adobe Photoshop). 
One might also imagine creativity support tools where abstraction-oriented tools and detail-oriented tools can be used in a complementary manner. Our study also showed that there is a time and place for both kinds of representations in a collaborative creative task, and that there is value in being able to move between abstract and concrete representations of a drawing. Rather than needing two separate creativity tools---one for drafting a prototype and one for generating a final creative work---future work might investigate how a tool could help a group seamlessly navigate between two views of the same work.

\subsection{Supporting Creative Collaboration}
Our studies were done at a relatively small scale; abstract blocks could also support structured, large-scale crowdsourcing workflows. One way people seek to collaborate with a crowd in creative work is through asking for feedback or guidance online through discussion forums or creative communities \cite{kuznetsov2010rise,settles2013,wasko2005should} like Behance (www.behance.com). Much of the exchange and communication occurs via text comment feedback and relies on the creator to interpret suggestions, make changes, and update their work. Methods for structuring communication and exchange among crowd collaborators include assessment \cite{Dow2012} and framing \cite{hicks2016framing} prompts, but much of creative work relies on being to \textit{see} and understand changes. In our observations, we saw that collaborators formed a consensus around sketch ideas through abstract blocking. Similar to bringing in examples as a way to show ``something like this'' \cite{kang2018paragon}, abstract blocks enable quick ways to visually demonstrate ideas without focusing on details.

We also observed that groups used abstract blocks as a way to delegate tasks among collaborators, using the blocks as structured components for each collaborator. At a larger scale, abstract blocking could aid in managing input from a large number of contributors. For example, in leader-facilitated crowdsourcing approaches such as flash teams, a leader could assign modular tasks to a team through blocks \cite{Retelny2014,Salehi2018,Valentine2017} or signal which parts of a creative work are open for collaboration \cite{kim2014ensemble}. In remixing communities, contributors might use blocking to build upon certain structural components of an existing project \cite{Resnick2009}. From a social perspective, abstract blocks can also give a tangible way for interactively exploring ideas, which could be particularly useful in creative livestreams where a crowd can engage with an artist and contribute ideas beyond text chat suggestions and audience voting games \cite{Fraser2019}. Future work should examine how abstract blocking might impact communication at a larger-scale across domains.

\subsection{Rich Abstraction or Simple Details?}
One limitation of using abstract blocks was a decrease in improvisation. Serendipitous creativity can occur through collaborators building and improvising off others' ideas \cite{Davis2017,Davis2016}. We saw this in the \textit{freeform} groups where groups would chain ideas together, adding on different details as the group discussed. Because \textit{abstraction} groups created a concrete composition plan ahead of drawing, we observed more expedient drawing, but with less improvisation overall. However, we did observe improvisation and exploration during the planning process when groups were first discussing sketch ideas. \textit{Abstraction} groups explored alternative composition ideas rather than details early on, reflecting the expert process we observed. One area for future work is examining what level of abstraction is appropriate at different stages of creative work. Could abstraction blocks be used to also specify concrete details? This could allow both for exploration of high-level compositions as well as exploration of lower-level details while removing the need for higher-fidelity sketching. Tools could provide different modes for different levels of abstraction, similar to multi-layered interfaces that adaptively disclose functionality as the user progresses \cite{shneiderman2002promoting}.

Another limitation to using abstract blocks is it is unclear \textit{how much} detail should be abstracted. Some groups in our study simply used text labels on blocks to communicate an idea. Others used lightly sketched images. One group used image search to provide a reference for their idea. Each of these representations might differentially impact communication and understanding of the ideas presented. In addition, most \textit{abstraction} groups did not discuss nuanced composition details such as the direction a character is facing or their position in the drawing. Instead, much of the discussion emphasized placement since the abstract blocks focused on adjusting the sketch's overall layout. We saw more of a discussion of these details in the \textit{freeform} group. 

Our observational design study presents several opportunities for how tools might implement abstraction strategies such as abstract blocking. The first is in supporting large-scale collaboration. We observed how tangible and malleable blocks helped groups communicate and iterate upon abstract, exploratory ideas. One question for future research is how abstract blocks might support communicating composition details while maintaining flexibility and high-level focus without compromising the benefits afforded by removing detail in the first place. Another opportunity is in comparing the use of abstraction blocks in individuals versus a group and seeing how exploration strategies might change based on the level of communication and collaboration. Lastly, we examined abstraction blocks in sketching, but the concept could apply to any domain where high-level goals can be broken into chunks. Future work should examine the nature of abstraction and how abstraction blocks might be utilized in domains outside of visual work.

\subsubsection{Summary}
In this chapter, we describe and evaluate using abstraction blocks, a technique where visual content is \emph{abstracted} to simple shapes to facilitate communication and collaboration on visual creative work such as drawings. By abstracting elements of visual work, we found in an observational design study that people are more likely to explore high-level concepts such as composition and perspective and are less likely to fixate on details of an early idea. Collaborative tasks rely on effective communication between contributors. For novices especially, communicating higher-level goals and changes can be challenging without domain or procedural knowledge. Abstraction blocks scaffold this communication, using simple shapes as a flexible medium to create conceptual chunks rather than detailed sketches. This abstraction makes the invisible visible and the intangible tangible. Collaborative creativity support systems, both for visual creative work and beyond, should include mechanisms for attuning users to the right level of abstraction to explore and communicate higher-level concepts.

\subsubsection{Acknowledgements}
We thank our research participants for their time and efforts. This research was funded in part by Adobe Research.

This chapter, in part, is being prepared for submission for publication by Tricia J. Ngoon, Joy O. Kim, and Scott Klemmer. The dissertation author was the primary investigator and author of this material.